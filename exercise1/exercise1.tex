\documentclass{article}
\usepackage{amsmath}


\begin{document}
\section*{Problem 1}
In the polar coordinates, we have 
\begin{align}
x =& r \sin \theta \cos \phi, \cr
y =& r \sin \theta \sin \phi, \cr
z =& r \cos \theta.
\end{align}
And 
\begin{align}
dx = & \sin \theta \cos \phi ~dr + r \cos \theta \cos \phi ~d\theta - r \sin \theta \sin \phi ~d\phi, \cr
dy = & \sin \theta \sin \phi ~dr + r \cos \theta \sin \phi ~d\theta+ r \sin \theta \cos \phi ~d\phi, \cr
dz = & \cos \theta ~dr - r \sin \theta ~d\theta.
\end{align}
Therefore,
\begin{align}
ds^2 =& dt^2 - dx^2 - dy^2 - dz^2\cr
=& dt^2 - (\sin \theta \cos \phi ~dr + r \cos \theta \cos \phi ~d\theta - r \sin \theta \sin \phi ~d\phi)^2\cr
&-(\sin \theta \sin \phi ~dr + r \cos \theta \sin \phi ~d\theta+ r \sin \theta \cos \phi ~d\phi)^2\cr
&-(\cos \theta ~dr - r \sin \theta ~d\theta)^2\cr
=& dt^2 - dr^2 - r^2~d\theta^2 - r^2 \sin^2\theta ~d\phi^2.
\end{align}
So, the induced metric,
\begin{equation}
g' = \begin{pmatrix}1 & 0 & 0 & 0\\ 0 & -1 & 0 & 0\\ 0 & 0 & -r^2 & 0\\ 0 & 0 & 0 & -r^2\sin^2\theta\end{pmatrix}.
\end{equation}


\section*{Problem 2}
\subsection*{1. Christoffel symbols are not components of a tensor}
Suppose we have two different local coordinates $\{x^\mu\}, \{y^\mu\}$ whose bases are $\{e_\mu\}=\{\partial/\partial x^\mu\}$ and $\{f_\mu\}=\{\partial/\partial y^\mu\}$ respectively. Denote the Christoffel symbols with respect to $y$-coordinates by $\widetilde{\Gamma}^\mu_{~\alpha\beta}$. The basis vector $f_\mu$ satisfies
\begin{equation}
\nabla_{f_\alpha} f_\beta = \widetilde{\Gamma}^\mu_{~\alpha \beta}f_\mu.
\end{equation}
If we write $f_\alpha = (\partial x^\sigma/ \partial y^\alpha)e_\sigma$, $f_\beta = (\partial x^\rho/ \partial y^\beta)e_\rho$, the LHS becomes
\begin{align}
\nabla_{f_\alpha} f_\beta =& \nabla_{f_\alpha}(\frac{\partial{x^\rho}}{\partial{y^\beta}}e_\rho)
=\frac{\partial^2 x^\rho}{\partial y^\alpha \partial y^\beta}e_\rho+\frac{\partial x^\sigma}{\partial y^\alpha}\frac{\partial x^\rho}{\partial y^\beta}\nabla_{e_\sigma}e_\rho\cr
=&\left(\frac{\partial^2 x^\nu}{\partial y^\alpha \partial y^\beta} + \frac{\partial x^\sigma}{\partial y^\alpha}\frac{\partial x^\rho}{\partial y^\beta} \Gamma^\nu_{~\sigma \rho} \right)e_\nu.
\end{align}
Since the RHS of (5) is equal to $\widetilde{\Gamma}^\mu_{~\alpha\beta} (\partial x^\nu / \partial y^\mu)e_\nu$, the Christoffel symbols must transform as
\begin{equation}
\widetilde{\Gamma}^\mu_{~\alpha\beta} = \frac{\partial x^\sigma}{\partial y^\alpha}\frac{\partial x^\rho}{\partial y^\beta} \frac{\partial y^\mu}{\partial x^\nu}\Gamma^\nu_{~\sigma \rho}
+\frac{\partial^2 x^\nu}{\partial y^\alpha \partial y^\beta} \frac{\partial y^\mu}{\partial x^\nu}.\end{equation}
Therefore, Christoffel symbols are not components of a tensor.

\subsection*{2. Christoffel symbols of Minkowski metric vanish}
Since the Christoffel symbols only involve the derivatives of the metric. The Minkowski metric is a constant metric, so its Christoffel symbols vanish.

\subsection*{3. Christoffel symbols in polar coordinates}
Recall the definition of the Christoffel symbols,
\begin{equation}
\Gamma^\mu_{~\alpha\beta} = \frac{1}{2} g^{\mu \nu}
\left(
\frac{\partial g_{\nu\alpha}}{\partial x^\beta} 
+ \frac{\partial g_{\beta \nu}}{\partial x^\alpha}
- \frac{\partial g_{\alpha \beta}}{\partial x^\nu}
\right).
\end{equation}
Using metric in (4), after some calculation, we can easily get,
\begin{align}
\Gamma^r_{~\theta\theta} = -r,  \quad \Gamma^r_{~\phi \phi} = -r \sin^2 \theta, \quad \Gamma^\theta_{~\phi \phi} = -\sin \theta \cos \theta,\cr
\Gamma^\theta_{~r\theta}=\Gamma^\theta_{~\theta r} = \frac{1}{r}, \quad
\Gamma^\phi_{~r \phi}= \Gamma^\phi_{~\phi r} = \frac{1}{r}, \quad
\Gamma^\phi_{~\theta \phi} = \Gamma^\phi_{~\phi \theta}=\cot \theta. 
\end{align}



\section*{Problem 3}
The metric in polar coordinates is given by
\begin{equation}
g_{\mu\nu} = 
\begin{pmatrix}
e^{N(r)} & 0 & 0 & 0 \\
0 & e^{-L(r)} & 0 & 0 \\
0 & 0 & -r^2 & 0 \\
0 & 0 & 0 & -r^2 \sin^2 \theta
\end{pmatrix}.
\end{equation}
Its inverse
\begin{equation}
g^{\mu\nu} = 
\begin{pmatrix}
e^{-N(r)} & 0 & 0 & 0 \\
0 & e^{L(r)} & 0 & 0 \\
0 & 0 & -\frac{1}{r^2} & 0 \\
0 & 0 & 0 & -\frac{1}{r^2 \sin^2 \theta}
\end{pmatrix}.
\end{equation}
After some calculation, we can get the Christoffel symbols as
\begin{align}
\Gamma^r_{~tt}=-\frac{1}{2} e^{N(r) + L(r)}N'(r), \quad 
\Gamma^r_{~rr}=-\frac{1}{2} L'(r), \quad
\Gamma^r_{~\theta\theta} = r e^{L(r)}, \cr
\Gamma^r_{~\phi\phi}= r \sin^2\theta e^{L(r)}, \quad
\Gamma^\theta_{~\phi\phi} = -\sin\theta \cos \theta, \quad
\Gamma^t_{~tr} = \Gamma^t_{~rt}=\frac{1}{2}N'(r),\cr
\Gamma^\theta_{~r \theta}=\Gamma^\theta_{~\theta r}=\frac{1}{r},\quad
\Gamma^\phi_{~r\phi}=\Gamma^\phi_{~\phi r} = \frac{1}{r},\quad
\Gamma^\phi_{~\theta \phi}=\Gamma^\phi_{~\phi \theta}=\cot \theta.
\end{align}


\section*{Problem 4}
\subsection*{1. covariant derivative of vectors}
The covariant derivative of a vector is defined as 
\begin{equation}
A^\mu_{~;\beta} = \frac{\partial A^\mu}{\partial x^\beta}+\Gamma^\mu_{~\alpha \beta}A^\alpha.
\end{equation}
Suppose we have two different local coordinates $\{x^\mu\}$, $\{y^\mu\}$. Denote the vector components and the Christoffel symbols with respect to $y$-coordinates by $\widetilde{A}^\mu$ and $\widetilde{\Gamma}^\mu_{~\alpha \beta}$. They transform as
\begin{equation}
\widetilde{A}^\mu = \frac{\partial y^\mu}{\partial x^\nu} A^\nu,\quad
\widetilde{\Gamma}^\mu_{~\alpha\beta} = \frac{\partial x^\sigma}{\partial y^\alpha}\frac{\partial x^\rho}{\partial y^\beta} \frac{\partial y^\mu}{\partial x^\nu}\Gamma^\nu_{~\sigma \rho}
+\frac{\partial^2 x^\nu}{\partial y^\alpha \partial y^\beta} \frac{\partial y^\mu}{\partial x^\nu}.
\end{equation}
First, we derive an alternative equation for the transformation law of Christoffel symbols. Using chain rule,
\begin{equation}
\frac{\partial y^\mu}{\partial x^\nu}\frac{\partial x^\nu}{\partial y^\alpha} = \frac{\partial y^\mu}{\partial y^\alpha}= \delta^\mu_\alpha.
\end{equation}
Since the Kronecker delta doesn't depend on the local coordinate.
We have
\begin{equation}
\frac{\partial}{\partial y^\beta}\left(\frac{\partial y^\mu}{\partial x^\nu}\frac{\partial x^\nu}{\partial y^\alpha} \right)=0.
\end{equation}
Now using Leibniz rule,
\begin{equation}
\frac{\partial^2 y^\mu}{\partial y^\beta\partial x^\nu}\frac{\partial x^\nu}{\partial y^\alpha}
+ \frac{\partial y^\mu}{\partial x^\nu}\frac{\partial^2 x^\nu}{\partial y^\alpha \partial y^\beta}=0,
\end{equation}
or 
\begin{equation}
\frac{\partial y^\mu}{\partial x^\nu}\frac{\partial^2 x^\nu}{\partial y^\alpha \partial y^\beta}=
-\frac{\partial^2 y^\mu}{\partial y^\beta\partial x^\nu}\frac{\partial x^\nu}{\partial y^\alpha}=-\frac{\partial x^\rho}{\partial y^\beta}\frac{\partial^2 y^\mu}{\partial x^\rho\partial x^\nu}\frac{\partial x^\nu}{\partial y^\alpha}.
\end{equation}
Therefore, the Christoffel symbols transform as
\begin{equation}
\widetilde{\Gamma}^\mu_{~\alpha\beta} = \frac{\partial x^\sigma}{\partial y^\alpha}\frac{\partial x^\rho}{\partial y^\beta} \frac{\partial y^\mu}{\partial x^\nu}\Gamma^\nu_{~\sigma \rho}
-\frac{\partial x^\rho}{\partial y^\beta}\frac{\partial^2 y^\mu}{\partial x^\rho\partial x^\nu}\frac{\partial x^\nu}{\partial y^\alpha}.
\end{equation}
Now, the ordinary derivative transforms as 
\begin{align}
\frac{\partial \widetilde{A}^\mu}{\partial y^\beta} =& \frac{\partial x^\rho}{\partial y^\beta}\frac{\partial}{\partial x^\rho}\left(\frac{\partial y^\mu}{\partial x^\nu} A^\nu\right)\cr
=&\left(\frac{\partial x^\rho}{\partial y^\beta}\frac{\partial y^\mu}{\partial x^\nu}\right)\frac{\partial A^\nu}{\partial x^\rho}
+\frac{\partial x^\rho}{\partial y^\beta}\frac{\partial^2 y^\mu}{\partial x^\rho \partial x^\nu} A^\nu.
\end{align}
The rest part transform as
\begin{align}
\widetilde{\Gamma}^\mu_{~\alpha\beta} \widetilde{A}^\alpha
=&\left(\frac{\partial x^\sigma}{\partial y^\alpha}\frac{\partial x^\rho}{\partial y^\beta} \frac{\partial y^\mu}{\partial x^\nu}\Gamma^\nu_{~\sigma \rho}
-\frac{\partial x^\rho}{\partial y^\beta}\frac{\partial^2 y^\mu}{\partial x^\rho\partial x^\nu}\frac{\partial x^\nu}{\partial y^\alpha}\right)
\left(\frac{\partial y^\alpha}{\partial x^\tau} A^\tau\right)\cr
=&\left(\frac{\partial x^\rho}{\partial y^\beta}\frac{\partial y^\mu}{\partial x^\nu}\right)\Gamma^\nu_{~ \sigma \rho}A^\sigma
-\frac{\partial x^\rho}{\partial y^\beta}\frac{\partial^2 y^\mu}{\partial x^\rho\partial x^\nu}A^\nu.
\end{align}
Adding them up yields
\begin{equation}
\frac{\partial \widetilde{A}^\mu}{\partial y^\beta} 
+\widetilde{\Gamma}^\mu_{~\alpha\beta} \widetilde{A}^\alpha 
=\left(\frac{\partial x^\rho}{\partial y^\beta}\frac{\partial y^\mu}{\partial x^\nu}\right)\left(\frac{\partial A^\nu}{\partial x^\rho}
+\Gamma^\nu_{~ \sigma \rho}A^\sigma\right).
\end{equation}
Therefore, the covariant derivative of vectors transform like tensors.

\subsection*{2. contravariant derivatives of vectors}
The contravariant derivative of a vector is defined as
\begin{equation}
B_{\mu;\beta}=\frac{\partial B_\mu}{\partial x^\beta}-\Gamma^\alpha_{~\mu \beta}B_\alpha.
\end{equation}
We have 
\begin{equation}
\widetilde{B}_\mu = \frac{\partial x^\nu}{\partial y^\mu}B_\nu,\quad
\widetilde{\Gamma}^\alpha_{~\mu\beta} = \frac{\partial x^\sigma}{\partial y^\mu}\frac{\partial x^\rho}{\partial y^\beta} \frac{\partial y^\alpha}{\partial x^\nu}\Gamma^\nu_{~\sigma \rho}
+\frac{\partial^2 x^\nu}{\partial y^\mu \partial y^\beta} \frac{\partial y^\alpha}{\partial x^\nu}.
\end{equation}
The ordinary derivative transforms as
\begin{align}
\frac{\partial \widetilde{B}_\mu}{\partial y^\beta}
=&\frac{\partial x^\rho}{\partial y^\beta}\frac{\partial}{\partial x^\rho}
\left(
\frac{\partial x^\sigma}{\partial y^\mu}B_\sigma
\right)\cr
=&\left(\frac{\partial x^\rho}{\partial y^\beta}\frac{\partial x^\sigma}{\partial y^\mu}\right)\frac{\partial B_\sigma}{\partial x^\rho}
+\frac{\partial^2 x^\sigma}{\partial y^\beta \partial y^\mu}B_\sigma.
\end{align}
The rest part transforms as
\begin{align}
-\widetilde{\Gamma}^\alpha_{~\mu \beta}\widetilde{B}_\alpha
=&-\left(\frac{\partial x^\sigma}{\partial y^\mu}\frac{\partial x^\rho}{\partial y^\beta} \frac{\partial y^\alpha}{\partial x^\nu}\Gamma^\nu_{~\sigma \rho}
+\frac{\partial^2 x^\nu}{\partial y^\mu \partial y^\beta} \frac{\partial y^\alpha}{\partial x^\nu}\right)
\left(
\frac{\partial x^\tau}{\partial y^\alpha}B_\tau
\right)\cr
=&-\left(\frac{\partial x^\sigma}{\partial y^\mu}\frac{\partial x^\rho}{\partial y^\beta}\right)\Gamma^\nu_{~\sigma \rho}B_\nu
-\frac{\partial^2 x^\nu}{\partial y^\mu \partial y^\beta}B_\nu.
\end{align}
Adding them up yields
\begin{equation}
\frac{\partial \widetilde{B}_\mu}{\partial y^\beta}-\widetilde{\Gamma}^\alpha_{~\mu \beta}\widetilde{B}_\alpha
= \left(\frac{\partial x^\rho}{\partial y^\beta}\frac{\partial x^\sigma}{\partial y^\mu}\right)
\left(\frac{\partial B_\sigma}{\partial x^\rho}-\Gamma^\nu_{~\sigma \rho}B_\nu\right).
\end{equation}
Therefore, the contravariant derivatives of vectors also transform like tensors.



\section*{Problem 5}
\begin{align}
g^{\mu\nu}_{~~;\beta} =& \frac{\partial g^{\mu\nu}}{\partial x^\beta}
+\Gamma^\mu_{~\alpha\beta}g^{\alpha \nu}
+\Gamma^\nu_{~\alpha\beta}g^{\mu \alpha}\cr
= &  \frac{\partial g^{\mu\nu}}{\partial x^\beta}
+ \frac{1}{2} g^{\mu \rho}
\left(
\frac{\partial g_{\alpha \rho}}{\partial x^\beta}
+\frac{\partial g_{\rho \beta}}{\partial x^\alpha}
-\frac{\partial g_{\alpha \beta}}{\partial x^\rho}
\right) g^{\alpha \nu}\cr
&+\frac{1}{2} g^{\nu \rho}
\left(
\frac{\partial g_{\alpha \rho}}{\partial x^\beta}
+\frac{\partial g_{\rho \beta}}{\partial x^\alpha}
-\frac{\partial g_{\alpha \beta}}{\partial x^\rho}
\right) g^{\alpha \mu}.
\end{align}
Since $\alpha$ and $\rho$ are dumb indices, interchange them in the third term and obtain
\begin{align}
g^{\mu\nu}_{~~;\beta} 
= &  \frac{\partial g^{\mu\nu}}{\partial x^\beta}
+ \frac{1}{2} g^{\mu \rho}
\left(
\frac{\partial g_{\alpha \rho}}{\partial x^\beta}
+\frac{\partial g_{\rho \beta}}{\partial x^\alpha}
-\frac{\partial g_{\alpha \beta}}{\partial x^\rho}
\right) g^{\alpha \nu}\cr
&+\frac{1}{2} g^{\nu \alpha}
\left(
\frac{\partial g_{\rho \alpha}}{\partial x^\beta}
+\frac{\partial g_{\alpha \beta}}{\partial x^\rho}
-\frac{\partial g_{\rho\beta}}{\partial x^\alpha}
\right) g^{\rho \mu}\cr
=& \frac{\partial g^{\mu\nu}}{\partial x^\beta}
+ g^{\mu \rho} g^{\alpha \nu} \frac{\partial g_{\alpha \rho}}{\partial x^\beta}.
\end{align}
Since $\delta^\nu_\rho = g_{\rho \alpha} g^{\alpha \nu}$, we have
\begin{equation}
\frac{\partial}{\partial x^\beta}\left(g_{\rho \alpha} g^{\alpha \nu}\right)
=g^{\alpha \nu}\frac{\partial g_{\rho \alpha}}{\partial x^\beta}
+g_{\rho \alpha}\frac{\partial g^{\alpha \nu}}{\partial x^\beta}
=0.
\end{equation}
Therefore 
\begin{equation}
g^{\mu \nu}_{~~;\beta}=\frac{\partial g^{\mu \nu}}{\partial x^\beta}
-g^{\mu \rho}g_{\rho \alpha}\frac{\partial g^{\alpha \nu}}{\partial x^\beta}
=\frac{\partial g^{\mu \nu}}{\partial x^\beta}
-\delta^\mu_\alpha\frac{\partial g^{\alpha \nu}}{\partial x^\beta}
=0.
\end{equation}

Similarly, 
\begin{align}
g_{\mu \nu;\beta} =& \frac{\partial g_{\mu\nu}}{\partial x^\beta}
-\Gamma^\alpha_{~\mu\beta}g_{\alpha \nu}
-\Gamma^\alpha_{~\nu\beta}g_{\mu\alpha}\cr
=&\frac{\partial g_{\mu\nu}}{\partial x^\beta}
-\frac{1}{2}g^{\alpha \rho}
\left(
\frac{\partial g_{\rho \beta}}{\partial x^\mu}
+\frac{\partial g_{\mu \rho}}{\partial x^\beta}
-\frac{\partial g_{\mu\beta}}{\partial x^\rho}
\right)
g_{\alpha \nu}\cr
&-\frac{1}{2}g^{\alpha \rho}
\left(
\frac{\partial g_{\rho \beta}}{\partial x^\nu}
+\frac{\partial g_{\nu \rho}}{\partial x^\beta}
-\frac{\partial g_{\nu\beta}}{\partial x^\rho}
\right)
g_{\alpha \mu}\cr
=&\frac{\partial g_{\mu\nu}}{\partial x^\beta}
-\frac{1}{2}\delta^{\rho}_{ \nu}
\left(
\frac{\partial g_{\rho \beta}}{\partial x^\mu}
+\frac{\partial g_{\mu \rho}}{\partial x^\beta}
-\frac{\partial g_{\mu\beta}}{\partial x^\rho}
\right)\cr
&-\frac{1}{2}\delta^{\rho}_{\mu}
\left(
\frac{\partial g_{\rho \beta}}{\partial x^\nu}
+\frac{\partial g_{\nu \rho}}{\partial x^\beta}
-\frac{\partial g_{\nu\beta}}{\partial x^\rho}
\right)
=0.
\end{align}



\section*{Problem 6}
\subsection*{1. properties of Riemann tensor}
Riemann tensor is given by
\begin{equation}
R^\alpha_{~\beta \mu\nu} = \Gamma^\alpha_{~\beta \nu, \mu}
-\Gamma^\alpha_{~ \beta\mu,\nu}
+\Gamma^\sigma_{~\beta\nu}\Gamma^\alpha_{~\sigma\mu}
-\Gamma^\sigma_{~\beta \mu}\Gamma^\alpha_{~\sigma \nu}.
\end{equation}
We lower the first index
\begin{align}
R_{\alpha \beta \mu \nu}=& g_{\alpha \tau}R^\tau_{~\beta \mu \nu}\cr
=& g_{\alpha \tau}\Gamma^\tau_{~\beta \nu,\mu}
-g_{\alpha \tau}\Gamma^\tau_{~\beta \mu,\nu}
+g_{\alpha \tau}\Gamma^\sigma_{~\beta\nu}\Gamma^\tau_{~\sigma\mu}
-g_{\alpha \tau}\Gamma^\sigma_{~\beta \mu}\Gamma^\tau_{~\sigma \nu}.
\end{align}
Also, in a local inertial frame, the curvature can vanish, though its derivative doesn't vanish. So we can ignore the last two terms in (34) and further expand (34) as
\begin{align}
R_{\alpha \beta \mu \nu} =& \frac{1}{2}\partial_\mu(\partial_\beta g_{\alpha \nu}+ \partial_\nu g_{\beta\alpha}-\partial_\alpha g_{\beta \nu})
-\frac{1}{2}\partial_\nu(\partial_\beta g_{\alpha \mu}+ \partial_\mu g_{\beta\alpha}-\partial_\alpha g_{\beta \mu})\cr
=&\frac{1}{2}(\partial_\mu\partial_\beta g_{\alpha \nu}
-\partial_\mu \partial_\alpha g_{\beta \nu}-\partial_\nu\partial_\beta g_{\alpha \mu}+\partial_\nu\partial_\alpha g_{\beta \mu})\cr
\end{align}
Then, it's easy to see that 
\begin{align}
R_{\alpha \beta \mu \nu} = -R_{\alpha \beta \nu \mu},
\quad R_{\alpha\beta\mu\nu} = -R_{\beta\alpha\mu\nu}, \cr
R_{\alpha \beta \mu \nu}= R_{\mu \nu \alpha \beta},
\quad R_{\alpha \beta \mu \nu}+ R_{\alpha \mu \nu \beta}+R_{\alpha \nu \beta \mu}=0.
\end{align}

\subsection*{2. Independent components in 4-dimensional spacetime}
Consider the tensor $R_{\alpha \beta \mu \nu}$ in general $n$-dimensional spacetime, since the tensor is antisymmetric in the first two indices, then the choice of $\alpha$ and $\beta$ is limited to $n(n-1)/2$. Similarly, the choice of $\mu$ and $\nu$ is also $n(n-1)/2$. Since the Riemann tensor is symmetric under the exchange of the pair of indices. So we can calculate the independent components as
\begin{equation}
\frac{1}{2}\left(\frac{1}{2}n(n-1)\right)\left(\frac{1}{2}n(n-1)+1\right)
=\frac{1}{8}(n^4-2n^3 +3n^2-2n).
\end{equation}
However, we haven't considered the cyclic identity $R_{\alpha [\beta \mu \nu]}=0$. A consequence of this identity is that the totally antisymmetric part of the Riemann tensor vanishes 
\begin{equation}
R_{[\alpha \beta  \mu \nu]}=0.
\end{equation}
In fact, this equation, plus other identities, are enough to imply the cyclic identity. So we only need to consider this equation instead. The number of independent components of a totally antisymmetric 4-index tensor is $n(n-1)(n-2)(n-3)/4!$. So we are left with 
\begin{equation}
\frac{1}{8}(n^4-2n^3 +3n^2-2n) - \frac{1}{24}n(n-1)(n-2)(n-3)=\frac{1}{12}n^2(n^2-1)
\end{equation}
independent components of the Riemann tensor.
Now, plug in $n=4$ and we get 20 independent components in 4-dimensional spacetime.

\subsection*{3. Independent components in 2d and 3d}
Since we have obtained the formula for general $n$-dimensional spacetime, we can see that there are 6 independent components in 3d and only 1 independent component in 2d.


\section*{Problem 7}
For a covariant vector $B_\alpha$, the parallel transport from $P$ to $P_1$ is 
\begin{equation}
B_\alpha - \Gamma^\beta_{\alpha\mu}(P)B_\beta d\xi^\mu.
\end{equation}
The parallel transport form $P_1$ tp $P'$ is 
\begin{equation}
B_\alpha - \Gamma^\beta_{\alpha\mu}(P)B_\beta d\xi^\mu
-\Gamma^\sigma_{\alpha \nu}(P_1)(B_\sigma-\Gamma^\beta_{\alpha\mu}(P)B_\beta d\xi^\mu)d \zeta^\nu.
\end{equation}
Since  $\Gamma^\sigma_{\alpha \nu}(P_1) = \Gamma^\sigma_{\alpha \nu}(P) + \Gamma^\sigma_{\alpha \nu, \mu}(P)d \xi^\mu $,
neglecting third order terms, we obtain
\begin{equation}
B_\alpha - \Gamma^\beta_{\alpha\mu}B_\beta d\xi^\mu
-\Gamma^\sigma_{\alpha \nu}B_\sigma d\zeta^\nu
+\Gamma^\sigma_{\alpha \nu}\Gamma^\beta_{\alpha\mu}B_\beta d\xi^\mu d \zeta^\nu
- \Gamma^\sigma_{\alpha \nu, \mu}B_\sigma d \xi^\mu d \zeta^\nu.
\end{equation}
Repeating the same procedure, passing first through $P_2$ and then through $P_1$, we obtain
\begin{equation}
B_\alpha - \Gamma^\beta_{\alpha\mu}B_\beta d\zeta^\mu
-\Gamma^\sigma_{\alpha \nu}B_\sigma d\xi^\nu
+\Gamma^\sigma_{\alpha \nu}\Gamma^\beta_{\alpha\mu}B_\beta d\zeta^\mu d \xi^\nu
- \Gamma^\sigma_{\alpha \nu, \mu}B_\sigma d \zeta^\mu d \xi^\nu.
\end{equation}
Subtracting the two quantities, we find
\begin{equation}
(\Gamma^\beta_{\alpha \mu, \nu}
- \Gamma^\beta_{\alpha \nu, \mu}
+\Gamma^\sigma_{\alpha \nu}\Gamma^\beta_{\alpha\mu}
-\Gamma^\sigma_{\alpha \mu}\Gamma^\beta_{\alpha\nu}  )
B_\beta
d \xi^\mu d \zeta^\nu .
\end{equation}
Therefore,
\begin{equation}
\Delta B_\alpha = R^\beta_{~\alpha \nu \mu}B_\beta d \xi^\mu d \zeta^\nu = R_{\alpha~\mu\nu}^{~\beta}B_\beta d \xi^\mu d \zeta^\nu.
\end{equation}



\end{document}