\documentclass{article}
\usepackage{amsmath}

\begin{document}
\section*{Problem 1}
The metric is given by 
\begin{equation}
    ds^2 = (1 + 2V) dt^2 - dx^2 - dy^2 - dz^2.
\end{equation}
From the definition of the Christoffel symbols,
\begin{equation}
    \Gamma^\mu_{~\rho\sigma} = \frac{1}{2}g^{\mu \nu} 
    \left(
    \frac{\partial g_{\nu \sigma}}{\partial x^\rho} 
    + \frac{\partial g_{\rho \nu}}{\partial x^\sigma} 
    - \frac{\partial g_{\rho \sigma}}{\partial x^\nu}
    \right),
\end{equation}
we can easily calculate that, for $i=1,2,3$,
\begin{equation}
    \Gamma^i_{~00} = -\frac{1}{2} g^{ii} \frac{\partial g_{00}}{\partial x^i}
    =-\frac{1}{2}(-1)\partial_i(1+2V)
    =\partial_i V,
\end{equation}
\begin{equation}
    \Gamma^0_{~i0} = \frac{1}{2} g^{00} \frac{\partial g_{00}}{\partial x^i}
    =\frac{\partial_i V}{1+2V}.
\end{equation}
In the weak field limit, $V$ is very small and we have
\begin{equation}
    \Gamma^i_{~00} = \Gamma^0_{i0} = \partial_iV.
\end{equation}



\section*{Problem 2}
We have to show that 
\begin{equation} 
    \label{toprove}
    \Gamma^\mu_{~\mu \alpha} = (\log \sqrt{-g})_{,\alpha}.
\end{equation}
The determinant is given by
\begin{equation}
    g = \sum_\mu g_{\mu \nu} \Delta^{\mu \nu},
\end{equation}
where $\Delta^{\mu\nu}$ is the algebraic cofactor and the summation
is only summed over $\mu$. The inverse of $g_{\mu \nu}$ is related 
to the algebraic cofactor by
\begin{equation}
    g^{\mu \nu} = \frac{\Delta^{\mu \nu}}{g}.
\end{equation}
Therefore,
\begin{equation}
    \frac{\partial g}{\partial g_{\mu \nu}} = \Delta^{\mu \nu} = g g^{\mu \nu}.
\end{equation}
The R.H.S of (6) equals
\begin{equation}
    \frac{\partial \log \sqrt{-g}}{\partial g} \frac{\partial g}{\partial x^\alpha} 
    = \frac{1}{2g} \frac{\partial g}{\partial x^\alpha}
    = \frac{1}{2g} \frac{\partial g}{\partial g_{\mu \nu}}\frac{\partial g_{\mu \nu}}{\partial x^\alpha}
    =\frac{1}{2}g^{\mu \nu} \frac{\partial g_{\mu \nu}}{\partial x^\alpha}.
\end{equation}
By definition, the L.H.S of (6) is 
\begin{equation}
    \Gamma^\mu_{~\mu \alpha} = \frac{1}{2} g^{\mu \nu} 
    \left(
        \frac{\partial g_{\nu \alpha}}{\partial x^\mu} 
        + \frac{\partial g_{\mu \nu}}{\partial x^\alpha}
        -\frac{\partial g_{\mu \alpha}}{\partial x^\nu}
    \right)
    =\frac{1}{2}g^{\mu \nu} \frac{\partial g_{\mu \nu}}{\partial x^\alpha}.
\end{equation}
In the last step, the first term cancels the last term because both $\mu$, $\nu$ are
dummy indices and $g^{\mu \nu} = g^{\nu \mu}$. So we have proved,
\begin{equation}
    \Gamma^\mu_{~\mu \alpha} = \left(\log \sqrt{-g}\right)_{,\alpha}.
\end{equation}


\section*{Problem 3}


\section*{Problem 4}



\section*{Problem 5}


\end{document}