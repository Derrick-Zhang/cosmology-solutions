\documentclass{article}
\usepackage{amsmath}

\begin{document}
\section*{Problem 1}
The metric is given by 
\begin{equation}
    ds^2 = (1 + 2V) dt^2 - dx^2 - dy^2 - dz^2.
\end{equation}
From the definition of the Christoffel symbols,
\begin{equation}
    \Gamma^\mu_{~\rho\sigma} = \frac{1}{2}g^{\mu \nu} 
    \left(
    \frac{\partial g_{\nu \sigma}}{\partial x^\rho} 
    + \frac{\partial g_{\rho \nu}}{\partial x^\sigma} 
    - \frac{\partial g_{\rho \sigma}}{\partial x^\nu}
    \right),
\end{equation}
we can easily calculate that, for $i=1,2,3$,
\begin{equation}
    \Gamma^i_{~00} = -\frac{1}{2} g^{ii} \frac{\partial g_{00}}{\partial x^i}
    =-\frac{1}{2}(-1)\partial_i(1+2V)
    =\partial_i V,
\end{equation}
\begin{equation}
    \Gamma^0_{~i0} = \frac{1}{2} g^{00} \frac{\partial g_{00}}{\partial x^i}
    =\frac{\partial_i V}{1+2V}.
\end{equation}
In the weak field limit, $V$ is very small and we have
\begin{equation}
    \Gamma^i_{~00} = \Gamma^0_{i0} = \partial_iV.
\end{equation}



\section*{Problem 2}
We have to show that 
\begin{equation} 
    \label{toprove}
    \Gamma^\mu_{~\mu \alpha} = (\log \sqrt{-g})_{,\alpha}.
\end{equation}
The determinant $g$ of the matrix $g_{\mu \nu}$ is given by
\begin{equation}
    g = \sum_\mu g_{\mu \nu} \Delta^{\mu \nu},
\end{equation}
where $\Delta^{\mu\nu}$ is the algebraic cofactor and the summation
is only summed over $\mu$. The inverse of $g_{\mu \nu}$ is related 
to the algebraic cofactor by
\begin{equation}
    g^{\mu \nu} = \frac{\Delta^{\mu \nu}}{g}.
\end{equation}
Therefore,
\begin{equation}
    \frac{\partial g}{\partial g_{\mu \nu}} = \Delta^{\mu \nu} = g g^{\mu \nu}.
\end{equation}
The R.H.S of (6) equals
\begin{equation}
    \frac{\partial \log \sqrt{-g}}{\partial g} \frac{\partial g}{\partial x^\alpha} 
    = \frac{1}{2g} \frac{\partial g}{\partial x^\alpha}
    = \frac{1}{2g} \frac{\partial g}{\partial g_{\mu \nu}}\frac{\partial g_{\mu \nu}}{\partial x^\alpha}
    =\frac{1}{2}g^{\mu \nu} \frac{\partial g_{\mu \nu}}{\partial x^\alpha}.
\end{equation}
By definition, the L.H.S of (6) is 
\begin{equation}
    \Gamma^\mu_{~\mu \alpha} = \frac{1}{2} g^{\mu \nu} 
    \left(
        \frac{\partial g_{\nu \alpha}}{\partial x^\mu} 
        + \frac{\partial g_{\mu \nu}}{\partial x^\alpha}
        -\frac{\partial g_{\mu \alpha}}{\partial x^\nu}
    \right)
    =\frac{1}{2}g^{\mu \nu} \frac{\partial g_{\mu \nu}}{\partial x^\alpha}.
\end{equation}
In the last step, the first term cancels the last term because both $\mu$, $\nu$ are
dummy indices and $g^{\mu \nu} = g^{\nu \mu}$. So we have proved,
\begin{equation}
    \Gamma^\mu_{~\mu \alpha} = \left(\log \sqrt{-g}\right)_{,\alpha}.
\end{equation}


\section*{Problem 3}
The action corresponding to the matter field is given by
\begin{equation}
    S_{\rm scalar} = \int \sqrt{-g} d^4x \left(
        \frac{1}{2} g^{\mu \nu} \partial_\mu \phi \partial_\nu \phi
        -\frac{1}{2}m^2\phi^2
    \right).
\end{equation}
The variation of the action with respect to the metric is given by
\begin{equation}
    \frac{\delta S_{\rm scalar}}{\delta g^{\alpha \beta}}
    = \frac{\delta \sqrt{-g}}{\delta g^{\alpha \beta}} 
    \left(\frac{1}{2}g^{\mu \nu} \partial_\mu \phi \partial_\nu \phi
    -\frac{1}{2} m^2\phi^2\right)
    +\frac{1}{2}\sqrt{-g}\partial_\alpha \phi \partial_\beta \phi.
\end{equation}
From equation (9) in the last problem, we have
\begin{equation}
    \delta g = g g^{\mu \nu} \delta g_{\mu \nu} = - g g_{\mu \nu}\delta g^{\mu \nu}.
\end{equation}
Therefore 
\begin{equation}
    \frac{\delta \sqrt{-g}}{\delta g^{\alpha \beta}}
    = -\frac{1}{2} \frac{1}{\sqrt{-g}} \frac{\delta g}{\delta g^{\alpha \beta}}
    = -\frac{1}{2} \sqrt{-g} g_{\alpha \beta},
\end{equation}
and the stress energy tensor is given by
\begin{equation}
   T_{\alpha \beta}=\frac{2}{\sqrt{-g}} \frac{\delta S_{\rm scalar}}{\delta g^{\alpha \beta}}
   = \partial_\alpha \phi \partial_\beta \phi 
   - \frac{1}{2} g_{\alpha \beta} g^{\mu \nu} \partial_\mu \phi \partial_\nu \phi
   + \frac{1}{2} g_{\alpha \beta} m^2 \phi^2.
\end{equation}
\section*{Problem 4}
We have to show that 
\begin{equation}
    \epsilon^{\mu \nu \rho \sigma} \equiv \frac{\varepsilon^{\mu \nu \rho \sigma}}{\sqrt{-g}},
\end{equation}
is a tensor.
First, consider a $4 \times 4$ matirx $M$, and from the definition of $\varepsilon^{\mu\nu\rho\sigma}$
and the definition of the determinant $\det M$, we have
\begin{equation}
    \varepsilon^{\mu \nu \rho \sigma} M^0_\mu M^1_\nu M^2_\rho M^3_\sigma = \det M,
\end{equation}
and further,
\begin{equation}
    \varepsilon^{\mu \nu \rho \sigma} M^\alpha_\mu M^\beta_\nu M^\gamma_\rho M^\delta_\sigma 
    = \varepsilon^{\alpha \beta \gamma \delta} \det M.
\end{equation}
Now suppose we have two different local coordinates $\{x^\mu\}$ and $\{y^\mu\}$,
and the Levi Civita symbol in those coordinates is denoted by $\varepsilon$ and $\varepsilon^\prime$,
metric denoted by $g$ and $g^\prime$.
We choose $M$ to be the Jacobian $(\partial y/ \partial x)$.
Then the above equation shows that
\begin{equation}
    \varepsilon^{\mu \nu \rho \sigma}\frac{\partial y^\alpha}{\partial x^\mu}
    \frac{\partial y^\beta}{\partial x^\nu}\frac{\partial y^\gamma}{\partial x^\rho}
    \frac{\partial y^\delta}{\partial x^\sigma} = \varepsilon^{\prime\alpha \beta \gamma \delta} \det\left(\frac{\partial y}{\partial x}\right).
\end{equation}
So $\varepsilon^{\mu \nu \rho \sigma}$ is a tensor density of weight $\omega = 1$. 
The metric transforms as
\begin{equation}
    g_{\mu \nu} = g^\prime_{\alpha \beta} \frac{\partial y^\alpha}{\partial x^\mu}
    \frac{\partial y^\beta}{\partial x^\nu}.
\end{equation}
Therefore, the quare root of the determinant $\sqrt{-g}$ transforms as
\begin{equation}
    \sqrt{-g} = \sqrt{-g^\prime} \det\left(\frac{\partial y}{\partial x}\right).
\end{equation}
So we have 
\begin{equation}
\frac{\varepsilon^{\prime\alpha \beta \gamma \delta}}{\sqrt{-g^\prime}}
= \frac{\varepsilon^{\mu \nu \rho \sigma}}{\sqrt{-g}}
\frac{\partial y^\alpha}{\partial x^\mu}
\frac{\partial y^\beta}{\partial x^\nu}\frac{\partial y^\gamma}{\partial x^\rho}
\frac{\partial y^\delta}{\partial x^\sigma},
\end{equation}
i.e.,
\begin{equation}
\epsilon^{\alpha \beta \gamma \delta}
= \epsilon^{\mu \nu \rho \sigma}
\frac{\partial y^\alpha}{\partial x^\mu}
\frac{\partial y^\beta}{\partial x^\nu}\frac{\partial y^\gamma}{\partial x^\rho}
\frac{\partial y^\delta}{\partial x^\sigma}.
\end{equation}
Therefore, $\epsilon^{\mu \nu \rho \sigma}$ is a tensor.



\section*{Problem 5}


\end{document}