\documentclass{article}
\usepackage{amsmath}

\begin{document}
\section*{Problem 1}
The metric is given by 
\begin{equation}
    ds^2 = d\theta^2 +\sin^2\theta d\phi^2.
\end{equation}
From the definition of the Christoffel symbols,
\begin{equation}
    \Gamma^\mu_{~\rho \sigma} = \frac{1}{2} g^{\mu \nu}
    \left(
        \frac{\partial g_{\nu \sigma}}{\partial x^\rho}
        +\frac{\partial g_{\rho \nu}}{\partial x^\sigma}
        -\frac{\partial g_{\rho \sigma}}{\partial x^\nu}
    \right),
\end{equation}
we can easily calculate that,
\begin{align}
    \Gamma^\theta_{~\phi \phi} =& -\frac{1}{2} g^{\theta \theta} 
    \frac{\partial g_{\phi \phi}}{\partial \theta}
    = -\frac{1}{2}\partial_\theta(\sin^2\theta)
    = -\sin \theta \cos \theta, \cr
    \Gamma^\phi_{~\theta \phi} = & \frac{1}{2} g^{\phi \phi} 
    \frac{\partial g_{\phi \phi}}{\partial \theta}
    = \frac{1}{2}\frac{1}{\sin^2\theta}\partial_\theta(\sin^2\theta)
    = \cot \theta.
\end{align}



\section*{Problem 2}
Given $ds^2 = dx^2 + dy^2 + dz^2$ and for a two-sphere with radius $R$,
we have
\begin{equation}
    \begin{cases}
        x = R \sin \theta \cos \phi\\
        y = R \sin \theta \sin \phi\\
        z = R \cos \theta
    \end{cases}.
\end{equation}
Then,
\begin{align}
    dx &= R \cos \theta \cos \phi ~d \theta - R \sin \theta \sin\phi ~d \phi,\cr
    dy &= R \cos \theta \sin \phi ~d \theta + R \sin \theta \cos \phi ~d \phi, \cr
    dz &= - R \sin \theta ~d \theta.
\end{align}
Therefore,
\begin{align}
    ds^2 =& dx^2 + dy^2 + dz^2\cr
    =& R^2 d\theta^2 + R^2 \sin^2 \theta d\phi^2.
\end{align}



\section*{Problem 3}
In the polar coordinates, we have $\vec x = r \hat r$ and
\begin{equation}
x = r \cos \theta, \quad y = r \sin \theta.
\end{equation}
Therefore,
\begin{equation}
    d \vec x = (dx, dy)
    = (\cos \theta dr - r\sin \theta d\theta,
    \sin \theta dr + r \cos \theta d\theta).
\end{equation}
And 
\begin{equation}
    d\vec x \cdot d \vec x = dx^2 + dy^2 = dr^2 + r^2 d \theta^2.
\end{equation}
Also
\begin{align}
   \vec x \cdot d \vec x =&~ r \cos^2 \theta dr 
   - r^2 \sin \theta \cos \theta d\theta
   + r \sin^2 \theta dr 
   + r^2 \sin \theta \cos \theta d\theta\cr
    =& ~r dr.
\end{align}



\section*{Problem 4}
Pove 
\begin{equation}
    (\xi^n)_\alpha^{~\beta} = K^n (x C x)^{n-1} C_{\alpha \rho} x^\rho x^\beta,
\end{equation}
by induction, where the matrix $\xi_\alpha^{~\nu} = K C_{\alpha \rho} x^\rho x^\nu$
and $xCx \equiv C_{\mu \nu} x^\mu x^\nu$.

First,
\begin{align}
    (\xi^2)_\alpha^{~\beta} = \xi_\alpha^{~\nu}\xi_\nu^{~\beta} 
    =&~ K^2 C_{\alpha \rho} x^\rho x^\nu C_{\nu \sigma} x^\sigma x^\beta \cr
    =&~ K^2 (x C x) C_{\alpha \rho} x^\rho x^\beta.
\end{align}
Suppose equation (11) holds for $n-1$,
\begin{equation}
    (\xi^{n-1})_\alpha^{~\beta} = K^{n-1} (x C x)^{n-2} C_{\alpha \rho} x^\rho x^\beta,
\end{equation}
then,
\begin{align}
    (\xi^n)_\alpha^{~\beta} = (\xi ^{n-1})_\alpha^{~\nu} \xi_\nu^{~\beta}
    =&~ (K^{n-1} (x C x)^{n-2} C_{\alpha \rho} x^\rho x^\nu)
    (K C_{\nu \sigma} x^\sigma x^\beta)\cr
    =&~ K^n (x C x)^{n-1} C_{\alpha \rho} x^\rho x^\beta.
\end{align}



\section*{Problem 5}
Prove that 
\begin{equation}
    R_{\mu \rho} = \partial_\mu \partial_\rho \log \sqrt{-g}
    -\frac{1}{\sqrt{-g}}\partial_\sigma(\sqrt{-g}~ \Gamma^\sigma_{~\mu \rho})
    + \Gamma^{\sigma}_{~\mu \alpha} \Gamma^\alpha_{~\rho \sigma}.
\end{equation}
From the definition of Ricci tensor, L.H.S. equals
\begin{equation}
    R_{\mu \rho} = R^\alpha_{~\mu \rho \alpha}
    =\Gamma_{~\mu \alpha, \rho}^{\alpha}
    -\Gamma_{~\mu \rho, \alpha}^{\alpha}
    +\Gamma_{~\mu \alpha}^{\sigma} \Gamma_{~\sigma \rho}^{\alpha}
    -\Gamma_{~\mu \rho}^{\sigma} \Gamma_{~\sigma \alpha}^{\alpha}.
\end{equation}
Fisrt expand the Christoffel symbol of the form
\begin{equation}
    \Gamma^\alpha_{~\mu \alpha} = \frac{1}{2} g^{\alpha \beta}
    \left(
        \frac{\partial g_{\alpha \beta}}{\partial x^\mu}
        +\frac{\partial g_{\mu \beta}}{\partial x^\alpha}
        -\frac{\partial g_{\mu \alpha}}{\partial x^\beta}
    \right)
    =\frac{1}{2} g^{\alpha \beta} \frac{\partial g_{\alpha \beta}}{\partial x^\mu}.
\end{equation}
In the last step, the last two terms inside the bracket, when dotted with $g^{\alpha \beta}$,
gives zero, because they are antisymmetric in $\alpha$, $\beta$, while the metric $g^{\alpha \beta}$
is symmetric in $\alpha$, $\beta$.
Now consider 
\begin{align}
    \partial_\mu \partial_\rho \log \sqrt{-g} 
    =&~ \partial_\rho\left(\frac{\partial g_{\alpha \beta}}{\partial x^\mu}
    \frac{\partial g}{\partial g_{\alpha \beta}}\frac{\partial }{\partial g}
    (\log \sqrt{-g})\right)\cr
    =&~ \partial_\rho\left(\frac{\partial g_{\alpha \beta}}{\partial x^\mu}
    (g g^{\alpha \beta})\left(\frac{1}{2g}\right)\right)\cr
    =&~ \partial_\rho \left(\frac{1}{2} g^{\alpha \beta}
    \frac{\partial g_{\alpha \beta}}{\partial x^\mu}\right)
    = \Gamma^\alpha_{~\mu \alpha, \rho}.
\end{align}
Next consider 
\begin{align}
    -\frac{1}{\sqrt{-g}} \partial_\sigma(\sqrt{-g} ~\Gamma^\sigma_{~\mu \rho})
    =&~ -\frac{1}{\sqrt{-g}} \partial_\sigma(\sqrt{-g}) \Gamma^\sigma_{~\mu \rho}
    -\Gamma^\sigma_{~\mu \rho, \sigma} \cr
    =&~ -\frac{1}{2} g^{\alpha \beta} \frac{\partial g_{\alpha \beta}}{\partial x^\sigma}
    \Gamma^\sigma_{~\mu \rho} - \Gamma^\alpha_{~\mu \rho, \alpha}\cr
    =&~ -\Gamma^\alpha_{~\sigma\alpha} \Gamma^\sigma_{~\mu \rho} - 
    \Gamma^\alpha_{~\mu\rho,\alpha}.
\end{align}
Therefore, R.H.S. of (15) equals
\begin{equation}
    \Gamma^\alpha_{~\mu \alpha, \rho} - \Gamma^\alpha_{~\mu \rho, \alpha}
    + \Gamma^\sigma_{~\mu \alpha} \Gamma^\alpha_{~\rho \sigma} 
    - \Gamma^\alpha_{~\sigma \alpha} \Gamma^\sigma_{~\mu \rho}.
\end{equation}
Compare (16) and (20), we see that equation (15) is proved.
\end{document}