\documentclass{article}
\usepackage{amsmath}

\begin{document}
\section*{Problem 1}
The metric is given by 
\begin{equation}
    ds^2 = d\theta^2 +\sin^2\theta d\phi^2.
\end{equation}
From the definition of the Christoffel symbols,
\begin{equation}
    \Gamma^\mu_{~\rho \sigma} = \frac{1}{2} g^{\mu \nu}
    \left(
        \frac{\partial g_{\nu \sigma}}{\partial x^\rho}
        +\frac{\partial g_{\rho \nu}}{\partial x^\sigma}
        -\frac{\partial g_{\rho \sigma}}{\partial x^\nu}
    \right),
\end{equation}
we can easily calculate that,
\begin{align}
    \Gamma^\theta_{~\phi \phi} =& -\frac{1}{2} g^{\theta \theta} 
    \frac{\partial g_{\phi \phi}}{\partial \theta}
    = -\frac{1}{2}\partial_\theta(\sin^2\theta)
    = -\sin \theta \cos \theta, \cr
    \Gamma^\phi_{~\theta \phi} = & \frac{1}{2} g^{\phi \phi} 
    \frac{\partial g_{\phi \phi}}{\partial \theta}
    = \frac{1}{2}\frac{1}{\sin^2\theta}\partial_\theta(\sin^2\theta)
    = \cot \theta.
\end{align}



\section*{Problem 2}
Given $ds^2 = dx^2 + dy^2 + dz^2$ and for a two-sphere with radius $R$,
we have
\begin{equation}
    \begin{cases}
        x = R \sin \theta \cos \phi\\
        y = R \sin \theta \sin \phi\\
        z = R \cos \theta
    \end{cases}.
\end{equation}
Then,
\begin{align}
    dx &= R \cos \theta \cos \phi ~d \theta - R \sin \theta \sin\phi ~d \phi,\cr
    dy &= R \cos \theta \sin \phi ~d \theta + R \sin \theta \cos \phi ~d \phi, \cr
    dz &= - R \sin \theta ~d \theta.
\end{align}
Therefore,
\begin{align}
    ds^2 =& dx^2 + dy^2 + dz^2\cr
    =& R^2 d\theta^2 + R^2 \sin^2 \theta d\phi^2.
\end{align}


\section*{Problem 3}
In the polar coordinates, we have $\vec x = r \hat r$ and
\begin{equation}
x = r \cos \theta, \quad y = r \sin \theta.
\end{equation}
Therefore,
\begin{equation}
    d \vec x = (dx, dy)
    = (\cos \theta dr - r\sin \theta d\theta,
    \sin \theta dr + r \cos \theta d\theta).
\end{equation}
And 
\begin{equation}
    d\vec x \cdot d \vec x = dx^2 + dy^2 = dr^2 + r^2 d \theta^2.
\end{equation}
Also
\begin{align}
   \vec x \cdot d \vec x =&~ r \cos^2 \theta dr 
   - r^2 \sin \theta \cos \theta d\theta
   + r \sin^2 \theta dr 
   + r^2 \sin \theta \cos \theta d\theta\cr
    =& ~r dr.
\end{align}


\end{document}